\documentclass{extarticle}
\sloppy

%%%%%%%%%%%%%%%%%%%%%%%%%%%%%%%%%%%%%%%%%%%%%%%%%%%%%%%%%%%%%%%%%%%%%%
% PACKAGES            																						  %
%%%%%%%%%%%%%%%%%%%%%%%%%%%%%%%%%%%%%%%%%%%%%%%%%%%%%%%%%%%%%%%%%%%%%
\usepackage[10pt]{extsizes}
\usepackage{amsfonts}
\usepackage{amsthm}
\usepackage{amssymb}
\usepackage[shortlabels]{enumitem}
\usepackage{microtype} 
\usepackage{amsmath}
\usepackage{mathtools}
\usepackage{commath}

%%%%%%%%%%%%%%%%%%%%%%%%%%%%%%%%%%%%%%%%%%%%%%%%%%%%%%%%%%%%%%%%%%%%%%
% PROBLEM ENVIRONMENT         																			           %
%%%%%%%%%%%%%%%%%%%%%%%%%%%%%%%%%%%%%%%%%%%%%%%%%%%%%%%%%%%%%%%%%%%%%
\usepackage{tcolorbox}
\tcbuselibrary{theorems, breakable, skins}
\newtcbtheorem{prob}% environment name
              {Problem}% Title text
  {enhanced, % tcolorbox styles
  attach boxed title to top left={xshift = 4mm, yshift=-2mm},
  colback=blue!5, colframe=black, colbacktitle=blue!3, coltitle=black,
  boxed title style={size=small,colframe=gray},
  fonttitle=\bfseries,
  separator sign none
  }%
  {} 
\newenvironment{problem}[1]{\begin{prob*}{#1}{}}{\end{prob*}}

%%%%%%%%%%%%%%%%%%%%%%%%%%%%%%%%%%%%%%%%%%%%%%%%%%%%%%%%%%%%%%%%%%%%%%
% THEOREMS/LEMMAS/ETC.         																			  %
%%%%%%%%%%%%%%%%%%%%%%%%%%%%%%%%%%%%%%%%%%%%%%%%%%%%%%%%%%%%%%%%%%%%%%
\newtheorem{thm}{Theorem}
\newtheorem*{thm-non}{Theorem}
\newtheorem{lemma}[thm]{Lemma}
\newtheorem{corollary}[thm]{Corollary}

%%%%%%%%%%%%%%%%%%%%%%%%%%%%%%%%%%%%%%%%%%%%%%%%%%%%%%%%%%%%%%%%%%%%%%
% MY COMMANDS   																						  %
%%%%%%%%%%%%%%%%%%%%%%%%%%%%%%%%%%%%%%%%%%%%%%%%%%%%%%%%%%%%%%%%%%%%%
\newcommand{\Z}{\mathbb{Z}}
\newcommand{\R}{\mathbb{R}}
\newcommand{\C}{\mathbb{C}}
\newcommand{\F}{\mathbb{F}}
\newcommand{\bigO}{\mathcal{O}}
\newcommand{\Real}{\mathcal{Re}}
\newcommand{\poly}{\mathcal{P}}
\newcommand{\mat}{\mathcal{M}}
\DeclareMathOperator{\Span}{span}
\newcommand{\Hom}{\mathcal{L}}
\DeclareMathOperator{\Null}{null}
\DeclareMathOperator{\Range}{range}
\newcommand{\defeq}{\vcentcolon=}
\newcommand\widebar[1]{\mathop{\overline{#1}}}
\newcommand{\restr}[1]{|_{#1}}
\DeclarePairedDelimiterX{\inp}[2]{\langle}{\rangle}{#1, #2}
\DeclarePairedDelimiter\Mod{\lvert}{\rvert}
\DeclarePairedDelimiter\Norm{\lVert}{\rVert}


%%%%%%%%%%%%%%%%%%%%%%%%%%%%%%%%%%%%%%%%%%%%%%%%%%%%%%%%%%%%%%%%%%%%%%
% SECTION NUMBERING																				           %
%%%%%%%%%%%%%%%%%%%%%%%%%%%%%%%%%%%%%%%%%%%%%%%%%%%%%%%%%%%%%%%%%%%%%%
\renewcommand\thesection{\Alph{section}:}



%%%%%%%%%%%%%%%%%%%%%%%%%%%%%%%%%%%%%%%%%%%%%%%%%%%%%%%%%%%%%%%%%%%%%%
% DOCUMENT START              																			           %
%%%%%%%%%%%%%%%%%%%%%%%%%%%%%%%%%%%%%%%%%%%%%%%%%%%%%%%%%%%%%%%%%%%%%%
\title{\vspace{-2em}Chapter 6: Inner Product Spaces}
\author{\emph{Linear Algebra Done Right}, by Sheldon Axler}
\date{}

\begin{document}
\maketitle



%%%%%%%%%%%%%%%%%%%%%%%%%%%%%%%%%%%%%%%%%%%%%%%%%%%%%%%%%%%%%%%%%%%%%
% SECTION A            																			           
%%%%%%%%%%%%%%%%%%%%%%%%%%%%%%%%%%%%%%%%%%%%%%%%%%%%%%%%%%%%%%%%%%%%%
\section{Inner Products and Norms}

% Problem 1
\begin{problem}{1}
Show that the function that takes $\left((x_1,x_2), (y_1,y_2)\right)\in \R^2\times \R^2$ to $\Mod{x_1y_1} + \Mod{x_2y_2}$ is not an inner product on $\R^2$.
\end{problem}
\begin{proof}
Suppose it were.  First notice
\begin{align*}
\inp{(1, 1) + (-1, -1)}{(1, 1)} &= \inp{(0, 0)}{(1, 1)}\\
&= \Mod{0\cdot 1} + \Mod{0\cdot 1}\\
&= 0.
\end{align*}
Next, since inner products are additive in the first slot, we also have
\begin{align*}
\inp{(1, 1) + (-1, -1)}{(1, 1)} &= \inp{(1, 1)}{(1, 1)} + \inp{(-1, -1)}{(1, 1)}\\
&= \Mod{1\cdot 1} + \Mod{1 \cdot 1} + \Mod{(-1)\cdot 1} + \Mod{(-1) \cdot 1}\\
&= 4.
\end{align*}
But this implies $0 = 4$, a contradiction.  Hence we must conclude that the function does not in fact define an inner product.
\end{proof}

% Problem 3
\begin{problem}{3}
Suppose $\F=\R$ and $V\neq \{0\}$.  Replace the positivity condition (which states that $\inp{v}{v}\geq0$ for all $v\in V$) in the definition of an inner product (6.3) with the condition that $\inp{v}{v} > 0$ for some $v\in V$.  Show that this change in the definition does not change the set of functions from $V\times V$ to $\R$ that are inner products on $V$.
\end{problem}
\begin{proof}
Let $V$ be a nontrivial vector space over $\R$, let $A$ denote the set of functions $V\times V\to\R$ that are inner products on $V$ in the standard definition, and let $B$ denote the set of functions $V\times V\to \R$ under the modified definition.  We will show $A = B$.\\
\indent Suppose $\inp{\cdot}{\cdot}_1\in A$.  Since $V\neq\{0\}$, there exists $v\in V-\{0\}$.  Then $\inp{v}{v}_1>0$, and so $\inp{\cdot}{\cdot}_1\in B$.  Thus $A\subseteq B$.\\ 
\indent Conversely, suppose $\inp{\cdot}{\cdot}_2 \in B$.  Then there exists some $v'\in V$ such that $\inp{v'}{v'}_2 > 0$.  Suppose by way of contradiction there exists $u\in V$ is such that $\inp{u}{u}_2 < 0$.  Define $w = \alpha u + (1- \alpha) v'$ for $\alpha\in\R$.  It follows
\begin{align*}
\inp{w}{w}_2 &= \inp{\alpha u + (1- \alpha) v'}{\alpha u + (1- \alpha) v'}_2\\
&= \inp{\alpha u}{\alpha u}_2 + 2\inp{\alpha u}{(1 - \alpha)v'}_2 + \inp{(1 - \alpha)v'}{(1 - \alpha)v'}_2\\
&= \alpha^2\inp{u}{u}_2 + 2\alpha(1-\alpha)\inp{u}{v'}_2 + (1-\alpha)^2\inp{v'}{v'}_2.
\end{align*}
Notice the final expression is a polynomial in the indeterminate $\alpha$, call it $p$.  Since $p(0) = \inp{v'}{v'}_2 > 0$ and $p(1) = \inp{u}{u}_2 < 0$, by Bolzano's theorem there exists $\alpha_0\in(0, 1)$ such that $p(\alpha_0) = 0$.  That is, if $ w = \alpha_0u + (1 - \alpha_0)v'$, then $\inp{w}{w}_2 = 0$.  In particular, notice $\alpha_0\neq 0$, for otherwise $w = v'$, a contradiction since $\inp{v'}{v'}_2 > 0$.  Now, since $\inp{w}{w}_2 = 0$ iff $w = 0$ (by the definiteness condition of an inner product), it follows 
\begin{equation*}
u = \frac{\alpha_0 - 1}{\alpha_0} v.
\end{equation*}
Letting $t =  \frac{\alpha_0 - 1}{\alpha_0}$, we now have
\begin{align*}
\inp{u}{u}_2 &= \inp{tv'}{tv'}\\
&= t^2\inp{v'}{v'}\\
&> 0,
\end{align*}
where the inequality follows since $t\in(-1, 0)$ and $\inp{v'}{v'}_2 > 0$.  This contradicts our assumption that $\inp{u}{u}_2 < 0$, and so we have $\inp{\cdot}{\cdot}_2\in A$.  Therefore, $B\subseteq A$.  Since we've already shown $A\subseteq B$, this implies $A = B$, as was to be shown.
\end{proof}

% Problem 5
\begin{problem}{5}
Let $V$ be finite-dimensional.  Suppose $T\in\Hom(V)$ is such that $\Norm{Tv}\leq \Norm{v}$ for every $v\in V$.  Prove that $T-\sqrt{2}I$ is invertible.
\end{problem}
\begin{proof}
Let $v\in\Null(T - \sqrt{2}I)$, and suppose by way of contradiction that $v\neq 0$.  Then
\begin{align*}
Tv - \sqrt{2}v = 0 &\implies Tv = \sqrt{2}v\\
&\implies \Norm{\sqrt{2}v}\leq \Norm{v}\\
&\implies \sqrt{2}\cdot \Norm{v}\leq \Norm{v}\\
&\implies \sqrt{2} \leq 1,
\end{align*}
a contradiction.  Hence $v = 0$ and $\Null(T - \sqrt{2}I) =\{0\}$, so that $T-\sqrt{2}I$ is injective.  Since $V$ is finite-dimensional, this implies $T-\sqrt{2}I$ is invertible, as desired.
\end{proof}

% Problem 7
\begin{problem}{7}
Suppose $u,v\in V$.  Prove that $\Norm{au + bv} = \Norm{bu + av}$ for all $a,b\in\R$ if and only if $\Norm{u} = \Norm{v}$.
\end{problem}
\begin{proof}

\end{proof}
\end{document}