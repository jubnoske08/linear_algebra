\documentclass{extarticle}
\sloppy

%%%%%%%%%%%%%%%%%%%%%%%%%%%%%%%%%%%%%%%%%%%%%%%%%%%%%%%%%%%%%%%%%%%%%%
% PACKAGES            																						  %
%%%%%%%%%%%%%%%%%%%%%%%%%%%%%%%%%%%%%%%%%%%%%%%%%%%%%%%%%%%%%%%%%%%%%
\usepackage[10pt]{extsizes}
\usepackage{amsfonts}
\usepackage{amsthm}
\usepackage{amssymb}
\usepackage[shortlabels]{enumitem}
\usepackage{microtype} 
\usepackage{amsmath}
\usepackage{mathtools}
\usepackage{commath}

%%%%%%%%%%%%%%%%%%%%%%%%%%%%%%%%%%%%%%%%%%%%%%%%%%%%%%%%%%%%%%%%%%%%%%
% PROBLEM ENVIRONMENT         																			           %
%%%%%%%%%%%%%%%%%%%%%%%%%%%%%%%%%%%%%%%%%%%%%%%%%%%%%%%%%%%%%%%%%%%%%
\usepackage{tcolorbox}
\tcbuselibrary{theorems, breakable, skins}
\newtcbtheorem{prob}% environment name
              {Problem}% Title text
  {enhanced, % tcolorbox styles
  attach boxed title to top left={xshift = 4mm, yshift=-2mm},
  colback=blue!5, colframe=black, colbacktitle=blue!3, coltitle=black,
  boxed title style={size=small,colframe=gray},
  fonttitle=\bfseries,
  separator sign none
  }%
  {} 
\newenvironment{problem}[1]{\begin{prob*}{#1}{}}{\end{prob*}}

%%%%%%%%%%%%%%%%%%%%%%%%%%%%%%%%%%%%%%%%%%%%%%%%%%%%%%%%%%%%%%%%%%%%%%
% THEOREMS/LEMMAS/ETC.         																			  %
%%%%%%%%%%%%%%%%%%%%%%%%%%%%%%%%%%%%%%%%%%%%%%%%%%%%%%%%%%%%%%%%%%%%%%
\newtheorem{thm}{Theorem}
\newtheorem*{thm-non}{Theorem}
\newtheorem{lemma}[thm]{Lemma}
\newtheorem{corollary}[thm]{Corollary}
\newtheorem*{definition}{Definition}

%%%%%%%%%%%%%%%%%%%%%%%%%%%%%%%%%%%%%%%%%%%%%%%%%%%%%%%%%%%%%%%%%%%%%%
% MY COMMANDS   																						  %
%%%%%%%%%%%%%%%%%%%%%%%%%%%%%%%%%%%%%%%%%%%%%%%%%%%%%%%%%%%%%%%%%%%%%
\newcommand{\Z}{\mathbb{Z}}
\newcommand{\R}{\mathbb{R}}
\newcommand{\C}{\mathbb{C}}
\newcommand{\F}{\mathbb{F}}
\newcommand{\bigO}{\mathcal{O}}
\newcommand{\Real}{\mathcal{Re}}
\newcommand{\poly}{\mathcal{P}}
\newcommand{\mat}{\mathcal{M}}
\DeclareMathOperator{\Span}{span}
\newcommand{\Hom}{\mathcal{L}}
\DeclareMathOperator{\Null}{null}
\DeclareMathOperator{\Range}{range}
\newcommand{\defeq}{\vcentcolon=}
\newcommand\widebar[1]{\mathop{\overline{#1}}}
\newcommand{\restr}[1]{|_{#1}}
\DeclarePairedDelimiterX{\inp}[2]{\langle}{\rangle}{#1, #2}
\DeclarePairedDelimiter\Mod{\lvert}{\rvert}
\DeclarePairedDelimiter\Norm{\lVert}{\rVert}


%%%%%%%%%%%%%%%%%%%%%%%%%%%%%%%%%%%%%%%%%%%%%%%%%%%%%%%%%%%%%%%%%%%%%%
% SECTION NUMBERING																				           %
%%%%%%%%%%%%%%%%%%%%%%%%%%%%%%%%%%%%%%%%%%%%%%%%%%%%%%%%%%%%%%%%%%%%%%
\renewcommand\thesection{\Alph{section}:}



%%%%%%%%%%%%%%%%%%%%%%%%%%%%%%%%%%%%%%%%%%%%%%%%%%%%%%%%%%%%%%%%%%%%%%
% DOCUMENT START              																			           %
%%%%%%%%%%%%%%%%%%%%%%%%%%%%%%%%%%%%%%%%%%%%%%%%%%%%%%%%%%%%%%%%%%%%%%
\title{\vspace{-2em}Chapter 7: Operators on Inner Product Spaces}
\author{\emph{Linear Algebra Done Right}, by Sheldon Axler}
\date{}

\begin{document}
\maketitle



%%%%%%%%%%%%%%%%%%%%%%%%%%%%%%%%%%%%%%%%%%%%%%%%%%%%%%%%%%%%%%%%%%%%%
% SECTION A            																			           
%%%%%%%%%%%%%%%%%%%%%%%%%%%%%%%%%%%%%%%%%%%%%%%%%%%%%%%%%%%%%%%%%%%%%
\section{Self-Adjoint and Normal Operators}

% Problem 1
\begin{problem}{1}
Suppose $n$ is a positive integer.  Define $T\in\Hom(\F^n)$ by
\begin{align*}
T(z_1, \dots, z_n) = (0, z_1, \dots, z_{n-1}).
\end{align*}
Find a formula for $T^\ast(z_1,\dots, z_n)$.
\end{problem}
\begin{proof}
Fix $(y_1,\dots, y_n)\in\F^n$.  Then for all $(z_1,\dots, z_n)\in \F^n$, we have
\begin{align*}
\inp{(z_1,\dots, z_n)}{T^\ast(y_1,\dots, y_n)} &= \inp{T(z_1,\dots, z_n)}{(y_1,\dots, y_n)}\\
&= \inp{(0, z_1,\dots, z_{n-1})}{(y_1,\dots, y_n)}\\
&= z_1y_2 + z_2y_3 + \dots + z_{n-1}y_n\\
&=\inp{(z_1,\dots, z_{n-1}, z_n)}{(y_2, \dots, y_n, 0)}.
\end{align*}
Thus $T^*$ is the left-shift operator.  That is, for all $(z_1,\dots, z_n)\in \F^n$, we have
\begin{align*}
T^\ast(z_1,\dots, z_n) = (z_2, \dots, z_n, 0),
\end{align*}
as desired.
\end{proof}

% Problem 2
\begin{problem}{2}
Suppose $T\in\Hom(V)$ and $\lambda\in\F$.  Prove that $\lambda$ is an eigenvalue of $T$ if and only if $\bar{\lambda}$ is an eigenvalue of $T^\ast$.
\end{problem}
\begin{proof}
Suppose $\lambda$ is an eigenvalue of $T$.  Then there exists $v\in V$ such that $Tv = \lambda v$.  It follows
\begin{align*}
\lambda \text{ is not an eigenvalue of }T &\iff T - \lambda I\text{ is invertible}\\
&\iff S(T - \lambda I) = (T- \lambda I)S = I \\
&~~~~~~~~~~\text{ for some }S\in\Hom(V)\\
&\iff S^\ast(T^\ast - \lambda I)^\ast = (T- \lambda I)^\ast S^\ast = I^\ast\\
&~~~~~~~~~~\text{ for some }S^\ast\in\Hom(V)\\
&\iff (T - \lambda I)^\ast\text{ is invertible}\\
&\iff T^\ast - \bar{\lambda}I \text{ is invertible}\\
&\iff \bar{\lambda} \text{ is not an eigenvalue of }T^\ast.
\end{align*}
Since the first statement and the last statement are equivalent, so too are their contrapositives.  Hence $\lambda$ is an eigenvalue of $T$ if and only if $\bar{\lambda}$ is an eigenvalue of $T^\ast$, as was to be shown.
\end{proof}

% Problem 3
\begin{problem}{3}
Suppose $T\in\Hom(V)$ and $U$ is a subspace of $V$.  Prove that $U$ is invariant under $T$ if and only if $U^\perp$ is invariant under $T^\ast$.
\end{problem}
\begin{proof}
$(\Rightarrow)$ First suppose $U$ is invariant under $T$, and let $x\in U^\perp$.  For any $u\in U$, it follows
\begin{align*}
\inp{T^\ast x}{u} &= \inp{x}{Tu}\\
&= 0,
\end{align*}
where the second equality follows since $Tu\in U$ (by hypothesis).  Thus $T^\ast x\in U^\perp$ for all $x\in U^\perp$.  That is, $U^\perp$ is invariant under $T^\ast$.\\
\indent $(\Leftarrow)$ Now suppose $U^\perp$ is invariant under $T^\ast$, and let $y\in U$.  For any $u'\in U^\perp$, it follows
\begin{align*}
\inp{T y}{u'} &= \inp{y}{T^\ast u'}\\
&= 0,
\end{align*}
where the second equality follows since $T^\ast u'\in U^\perp$ (by hypothesis).  Thus $T y\in U$ for all $y\in U$.  That is, $U$ is invariant under $T$, completing the proof.
\end{proof}

% Problem 5
\begin{problem}{5}
Prove that 
\begin{align*}
\dim\Null T^\ast = \dim\Null T + \dim W - \dim V
\end{align*}
and
\begin{align*}
\dim\Range T^* = \dim\Range T
\end{align*}
for every $T\in\Hom(V, W)$.
\end{problem}
\begin{proof}
Let $T\in\Hom(V, W)$.  Notice
\begin{align*}
\dim\Null T^\ast &= \dim\left(\Range T\right)^\perp\\
&=\dim W - \dim\Range T\\
&=\dim W + \dim\Null T - \dim V,
\end{align*}
where the first equality follows by 7.7(a), the second equality follows by 6.50, and the third equality follows by the Fundamental Theorem of Linear Maps.  Next notice
\begin{align*}
\dim\Range T^\ast &= \dim\left(\Null T\right)^\perp\\
&= \dim V - \dim\Null T\\
&= \dim \Range T,
\end{align*}
where the first equality follows by 7.7(b), and the second and third equalities follow again by the same theorems above.
\end{proof}

% Problem 7
\begin{problem}{7}
Suppose $S,T\in\Hom(V)$ are self-adjoint.  Prove that $ST$ is self-adjoint if and only if $ST=TS$.
\end{problem}
\begin{proof}
$(\Rightarrow)$ Suppose $ST$ is self-adjoint.  We have
\begin{align*}
ST &= (ST)^\ast\\
&= T^\ast S^\ast\\
&=TS,
\end{align*}
where the second equality follows by 7.6(e).\\
\indent $(\Leftarrow)$ Conversely, suppose $ST = TS$.  It follows
\begin{align*}
(ST)^\ast &= (TS)^\ast\\
&= S^\ast T^\ast,
\end{align*}
where the second equality again follows by 7.6(e), completing the proof.
\end{proof}

% Problem 9
\begin{problem}{9}
Suppose $V$ is a complex inner product space with $V\neq\{0\}$.  Show that the set of self-adjoint operators on $V$ is not a subspace of $\Hom(V)$.
\end{problem}
\begin{proof}
Let $\mathcal{A}$ denote the set of self-adjoint operators on $V$, and suppose $T\in\mathcal{A}$.  By 7.6(b), notice $(iT)^\ast = -iT^\ast$, so that $\mathcal{A}$ is not closed under scalar multiplication.  Thus $\mathcal{A}$ is not a subspace of $\Hom(V)$.
\end{proof}

% Problem 11
\begin{problem}{11}
Suppose $P\in\Hom(V)$ is such that $P^2 = P$.  Prove that there is a subspace $U$ of $V$ such that $P = P_U$ if and only if $P$ is self-adjoint.
\end{problem}
\begin{proof}
$(\Rightarrow)$ First suppose there is a subspace $U\subseteq V$ such that $P = P_U$, and let $v_1,v_2\in V$.  It follows
\begin{align*}
\inp{Pv_1}{v_2} &= \inp{u_1}{u_2 + w_2}\\
&= \inp{u_1}{u_2} + \inp{u_1}{w_2}\\
&= \inp{u_1}{u_2}\\
&= \inp{u_1}{u_2} + \inp{w_1}{u_2}\\
&= \inp{u_1 + w_1}{u_2}\\
&= \inp{v_1}{Pv_2},
\end{align*}
and thus $P = P^\ast$.\\
\indent $(\Leftarrow)$ Conversely, suppose $P = P^\ast$.  Let $v\in V$.  Notice $P(v - Pv)= Pv - P^2v = 0$, and hence $v - Pv\in\Null P$.  By 7.7(c), we know $\Null P = \left(\Range T^\ast\right)^\perp$.  By hypothesis, $P$ is self-adjoint, and hence we have $v-Pv \in \left(\Range T\right)^\perp$.  Notice we may write
\begin{align*}
v = Pv + (v - Pv),
\end{align*}
where $Pv \in \Range P$ and $v-Pv \in \left(\Range T\right)^\perp$.  Let $U = \Range P$.  Since the above holds for all $v\in V$, we conclude $P = P_U$, and the proof is complete.
\end{proof}

% Problem 13
\begin{problem}{13}
Give an example of an operator $T\in\Hom(\C^4)$ such that $T$ is normal but not self-adjoint.
\end{problem}
\begin{proof}
Let $T$ be the operator on $\F^2$ whose matrix with respect to the standard basis is
\begin{align*}
\begin{bmatrix}
2 & -3 & 0 & 0\\
3 &  2 & 0 & 0\\
0 &  0 & 0 & 0\\
0 &  0 & 0 & 0
\end{bmatrix}.
\end{align*} 
We claim $T$ is both normal and not self-adjoint.  To see that $T$ is not self-adjoint, notice that the entry in row $2$, column $1$ does not equal the complex conjugate of the entry in row $1$ column $2$.\\
\indent Next, notice
\begin{align*}
\mat(TT^\ast) &= 
\begin{bmatrix}
2 & -3 & 0 & 0\\
3 &  2 & 0 & 0\\
0 &  0 & 0 & 0\\
0 &  0 & 0 & 0
\end{bmatrix}
\begin{bmatrix}
2  & 3 & 0 & 0\\
-3 & 2 & 0 & 0\\
0  & 0 & 0 & 0\\
0  & 0 & 0 & 0
\end{bmatrix}
= 
\begin{bmatrix}
13  & 0 & 0 & 0\\
0 & 13 & 0 & 0\\
0  & 0 & 0 & 0\\
0  & 0 & 0 & 0
\end{bmatrix}
\end{align*}
and 
\begin{align*}
\mat(T^\ast T) &= 
\begin{bmatrix}
2 & 3 & 0 & 0\\
-3 &  2 & 0 & 0\\
0 &  0 & 0 & 0\\
0 &  0 & 0 & 0
\end{bmatrix}
\begin{bmatrix}
2  & -3 & 0 & 0\\
3 & 2 & 0 & 0\\
0  & 0 & 0 & 0\\
0  & 0 & 0 & 0
\end{bmatrix}
= 
\begin{bmatrix}
13  & 0 & 0 & 0\\
0 & 13 & 0 & 0\\
0  & 0 & 0 & 0\\
0  & 0 & 0 & 0
\end{bmatrix},
\end{align*}
and hence $TT^\ast$ and $T^\ast T$ have the same matrix.  Thus $TT^\ast = T^\ast T$, and $T$ is normal.   
\end{proof}

% Problem 15
\begin{problem}{15}
Fix $u,x\in V$.  Define $T\in\Hom(V)$ by
\begin{align*}
Tv = \inp{v}{u}x
\end{align*}
for every $v\in V$.
\begin{itemize}[(a)]
\item Suppose $\F = \R$.  Prove that $T$ is self-adjoint if and only if $u,x$ is linearly dependent.
\item Prove that $T$ is normal if and only if $u,x$ is linearly dependent.
\end{itemize}
\end{problem}
\begin{proof}
We first derive a useful formula for $T^\ast$ which we'll use in both (a) and (b).  Let $w_1,w_2\in V$ and notice
\begin{align*}
\inp{w_1}{T^\ast w_2} &= \inp{Tw_1}{w_2}\\
&= \inp{\inp{w_1}{u}x}{w_2}\\
&= \inp{w_1}{u}\inp{x}{w_2}\\
&= \inp{w_1}{\widebar{\inp{x}{w_2}}u}\\
&= \inp{w_1}{\inp{w_2}{x}u},
\end{align*}
and thus $T^\ast w_2 = \inp{w_2}{x}u$.  Since $w_2$ is arbitrary, we may rewrite this as $T^\ast v = \inp{v}{x}u$ for all $v\in V$.
\begin{itemize}[(a)]
\item $(\Rightarrow)$ Suppose $T$ is self-adjoint.  Then we have
\begin{align*}
\inp{v}{u}x - \inp{v}{x}u = Tv - T^\ast v = 0
\end{align*}
for all $v\in V$.  In particular, we have
\begin{align*}
\inp{u}{u}x - \inp{u}{x}u = 0.
\end{align*}
We may assume both $u$ and $x$ are nonzero (for otherwise there is nothing to prove).  Hence $\inp{u}{u}\neq 0$, which forces $\inp{u}{x}$ to be nonzero as well, and thus the equation above shows $u,x$ is linearly dependent.
\end{itemize}
\end{proof}
\end{document}